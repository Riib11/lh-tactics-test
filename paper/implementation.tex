\section{Implementation}

Organization idea:
\begin{enumerate}
  \item Overview the entire procedure:
  \begin{enumerate}
    \item The user writes a proof macro term in ``tactic'' quasiquotes as their proof, in EDSL1 i.e. the proof macro language
    \item The proof macro term is processed into a ``proto proof term'' in EDSL2 i.e. the metadata-augmented subset of Haskell relevant for extrinsic proofs
    \item The ``proto proof term'' is embedded into Haskell, then spliced the place of the original proof macro. The file is checked.
    \item If the check passes, then the proto proof term is pruned by repeatedly pruning the proto proof term, embedding and splicing it into the place of the original proof macro, and checking if the pruned result still passes.
    \item The resulting proof term is a valid proof as determined by Liquid Haskell.
  \end{enumerate}
  \item Detail the features of EDSL1 and the intended behaviors
  \item Briefly describe the details of EDSL2
  \begin{enumerate}
    \item Describe how each feature in EDSL1 is implemented by a processing transformation into EDSL2
    \item Preprocessing EDSL1 into EDSL2 is a contextual transformation, since it needs to keep track of an environment which includes various information (see Language.Core.Syntax.Environment'')
    \item How the ``auto'' term in EDSL2 includes metadata for which terms have have been pruned and which have been determined necessary.
  \end{enumerate}
  \item Detail pruning procedure
  \begin{enumerate}
    \item Clearly, this can be done more efficiently.
    \item Point out further possible work like checking for well-typed applications where the function has refined argument types.
  \end{enumerate}
\end{enumerate}

% 
% * real start
% 
  
\todo{is {process} the right term? or {preprocess}, or {expand}, or {elaborate}? something that describes converting a macro to its fully-concretized form}

\todo{figure out standard vocabularly for referring to different parts of macro system and LH:
--- a {declaration}? defines the name, type, and arguments of a top-level definition
--- a {proof macro declaration}? is the macro that processes to a definition
--- a {proof macro}? is a macro that goes in the sequence of macros in a {proof macro definition}
}


The proof macro system processes an input proof macro into an output Liquid Haskell proof into stages:
\begin{enumerate}
  \item 
  The user writes the input proof macro in Template Haskell quasiquotes. 
  Then the user runs \TheTool \todo{how to talk about this executable} on the file containing the target proof macro to process. 
  The quasiquoted proof macro is parsed into \LangA.
  \item
  The \LangATerm is processed into a corresponding \LangBTerm, and all metadata values are given initial defaults.
  \item
  The \LangBTerm is cached, embedded into Haskell, and then spliced in place of the original proof macro. 
  \item 
  The cached \LangBTerm is repeatedly pruned, using metadata to track pruning progress, where each pruning step involves removing some proof terms contained inside of the entire \LangBTerm, embedding and splicing it in place of the original proof macro, and then running Liquid Haskell to check that the prune step was valid.
\end{enumerate}
  
\subsection{The \LangA}

The \LangA defines a collection of proof macros that are meant to resemble Coq tactics.
The proof macro system is designed to be extendible, so that new proof macros can be added easily by adding a new constructor to the \LangA and then handling the new case for processing.
The syntax for \LangA is the following:

\begin{align*}
  \textit{decl-macro} ::= &
    ~ f ~ : ~ \textit{typ} \\ &
    ~ f ~ \overline{y_i} ~ = ~ \overline{\textit{exp-macro} ;}
  \\[1em]
  \textit{exp-macro} ::= &
    ~ \MC{induct} ~ x \\ \mid &
    ~ \MC{destruct} ~ \text{exp} \\ \mid &
    ~ \MC{assert} ~ \textit{exp} \\ \mid &
    ~ \MC{dismiss} ~ \textit{exp} \\ \mid &
    ~ \MC{condition} ~ \textit{exp} \\ \mid &
    ~ \MC{auto} ~ [\overline{\textit{x} ,}] ~ n \\ \mid &
    ~ \MC{use} ~ \textit{exp} \\ \mid &
    ~ \MC{trivial}
  \\[1em]
  \textit{exp} ::= & ~ \textit{Haskell expression} 
  \\
  \textit{typ} ::= & ~ \textit{Haskell type (monomorphic)} 
  \\
  f, x, y_i ::= & ~ \textit{Haskell name} 
  \\
  n \in & ~ \mathbb{N}
\end{align*}

There are two main types of proof macros: 
\begin{itemize}
  \item
  \textit{Control flow} macros are processed into control flow structures, such as pattern matching. 
  If such a macro yields a control flow structure that has mutliple branches, then a macro-processing branch is created for each of these branches, and the sequence of proof macros following it are processed in each of these branches.
  The control flow macros are:
  \begin{itemize}
    \item
    $\MC{induct} ~ x$ --- patterm matches on $x$, which must be a function argument in the declaration macro.
    In each case of the pattern match, the introduced variables are included in the recursion context at the corresponding argument position of $x$, and the original $x$ is removed from the context.
    This recursion context keeps track of what expressions are allowed to be given as an argument, in that argument position, to a recursion.
    A macro-processing branch is created for each case.
    \item 
    $\MC{destruct} ~ \textit{exp}$ --- pattern matches on $\textit{exp}$. A macro-processing branch is created for each case.
    \item $\MC{condition} ~ \textit{exp}$ --- conditions on $\text{exp}$. A macro-processing branch is created for the \MC{then} and \MC{else} branches respectively.
    \item $\MC{assert} ~ \textit{exp}$ --- conditions on $\text{exp}$. A macro-processing branch is created for the \MC{then} branch, but the \MC{else} branch is only filled with \MC{trivial}.
    \item $\MC{dismiss} ~ \textit{exp}$ --- conditions on $\text{exp}$. A macro-processing branch is created for the \MC{else} branch, but the \MC{then} branch is only filled with \MC{trivial}.
  \end{itemize}
  \item
  \todo{think of better name? perhaps \textit{evidence-providing}?}
  \textit{Evidence} macros are processed into terms that provide evidence to the Liquid Haskell checker, such as introducing a lemma to the refinement context.
  \todo{is this the right place to put this note?}
  Note that if a sequence of macros given in declaration macro (i.e. \textit{decl-macro}) does \textit{not} end in an evidence macro, then an \LC{auto} macro is included implicitly at the end of the sequence.
  The evidence macros are:
  \begin{itemize}
    \item
    $\MC{auto} ~ [\overline{\textit{x} ,}] ~ n$ --- generates all well-typed neutral forms that have type \LC{Proof}, up to height $n$, using variables in context and given as hints. 
    The height of a neutral form is the height of its AST, where applications are multi-ary rather than nested binary (e.g. \LC{f x y z} has height 2 rather than height 4).
    Recursive neutral forms are also generated in this way, but have the additional restriction that a recursive neutral form must have as one of its arguments a variable from that argument position's recursion context, ensuring that Liquid Haskell will be able to determine that the resulting Haskell function is terminating.
    Note that the hints and height arguments to this macro are optional, and take values \LC{[]} and \LC{3} respectively by default.
    \item $\MC{use} ~ \textit{exp}$ --- includes the refinement of the expression's type into the refinement context.
    \item $\MC{trivial}$ --- includes \LC{trivial :: Proof} into the resulting Haskell term. Note that $\MC{trivial} = \MC{()}$ and $\MC{Proof} = \MC{()}$.
  \end{itemize}
\end{itemize}

\todo{for sake a readability, I used the usual list notations such as [N] and (++). is this ok? since in the actual implementation, I decided not to overload the built-in Haskell list notations}
For example, the following is a proof macro for generating a proof of ``if \LC{x} is in a list \LC{xs}, then \LC{x} is also in the list \LC{xs ++ ys}'':
\begin{code}
  prop x xs ys = if elem x ys then elem x (xs ++ ys) else True

  {-@ elem_concat :: x:N -> xs:[N] -> ys:[N] -> {prop x xy ys} @-}
  [tactic|
  elem_concat :: N -> [N] -> [N] -> Proof
  elem_concat x xs ys =
    assert {elem x ys};
    induct xs
  |]
\end{code}

\subsection{The \LangB}

The \LangB is a subset of Haskell with some additional metadata.
The syntax of \LangB is the following:
% \begin{align*}
%   \textit{decl}~ ::= &
%     \textit{name} ~ : ~ \textit{type} \\ &
%     \textit{name} ~ \overline{\textit{name}} ~ = ~ \textit{expr}
%   \\
%   \textit{expr}~ ::= &
%     \mathbf{\lambda} ~ \textit{name} ~ \textbf{->} ~ \textit{instr} \\ &
%     \textbf{case} ~ \textit{exp} ~ \textbf{of} ~ \overline{\textit{pat} ~ \textbf{->} ~ \textit{exp} ;} \\ & 
%     \textbf{if} ~ \textit{exp} ~ \textbf{then} ~ \textit{exp} ~ \textbf{else} ~ \textit{exp} \\ & 
%     \textbf{auto} ~ [\overline{\textit{exp} ;}] \textit{pruning-metadata} \\ & 
%     \textbf{trivial}
% \end{align*}
%  
A term of \LangB can be directly embedded into Haskell by processing the \LC{Auto} structure in this way:
\todo{how to format this?}
\begin{code}
  Auto { init = [a1, ..., aM], init = [b1, ..., bN], pruned } ==>
  a1 &&& ... &&& aM &&& b1 &&& ... &&& bN
\end{code}

\todo{should I convert this to an algorithmic format? or should I just not include this algorithm's full detail at all}
A \LangA term is processed into an \LangB term via the following algorithm:
\begin{verbatim}
 processDecl : EDSL1-decl -> EDSL2-decl
 processDecl (name, args, type, instrs) =
   set function name to name
   set output type to type
   for each arg
     get output type as a -> b 
     store type of arg as a
     set output type to b
   process instrs
  
 // stateful
 process : [EDSL1-instr] -> EDSL2-expr
 process [] = []
 process (instr : instrs) = case instr of 
   intro x -> do
     [|a <- b|] <- get output type
     type of x := a
     output type := b
     [|\x -> $(process instrs)|]
   destruct exp requires xs -> do
     if all of xs are in scope then 
       varss <- get patterns for deconstructing exp
       constrs <- constructors of the datatype of exp
       matches <- 
         for each (vars, constr) in (varss * constrs)
           args <- argument types of constr
           for each (var, arg) in in (pats * args)
             type of var := arg
           [|ps -> $(process instrs)|]
       [|case exp of matches|]
     else
       process instrs
   induct exp requires xs -> do
     if all of xs are in scope then 
       varss <- get patterns for deconstructing exp
       constrs <- constructors of the datatype of exp
       matches <-
         for each (vars, constr) in (varss * constrs)
           args <- argument types of constr
           for each (var, arg) in in (pats * args)
             if exp is an argument to the top-level proposition then
               add var to context of recursive-safe expression to fill the
               respective argument of a recursive call
             type of var := arg
           [|ps -> $(process instrs)|]
       [|case exp of matches|]
     else
       process instrs
   auto hints depth ->
     generate all neutral forms in context, up to depth
     when a recursion is generated, at least one of its arguments must be something from the context of recursive safe expressions for that argument (which is populated via induct)
   assert exp requires xs ->
     if all of xs are in scope then
       [|if exp then $(process instrs) else True|]
     else
       process instrs
   dismiss exp requires xs ->
     if all of xs are in scope then
       [|if exp then True else $(process instrs)|]
     else
       process instrs
   use exp requires xs -> 
     [|use exp &&& $(process instrs)]
   condition exp requires xs ->
     [|if exp then $(process instrs) else $(process instrs)]
   trivial ->
     [|trivial|]
\end{verbatim}

\todo{how to properly reference this}
Recall the proof macro used to prove \LC{elem_concat}.
The macro declaration begins with a type signature and then \LC{elem_concat x xs ys}.
From the type signature, the proof macro system can deduce the types of \LC{x}, \LC{xs}, and \LC{ys} and bring these variables into context.
\footnote{Unfortunately, types of local bindings cannot be inferred wiht Template Haskell, so the type signature is in fact necessary.}
% TODO: (elem x ys) should be {elem x ys}, but causes {too many {}} error
The sequence of proof macros begins with \LC{assert (elem x ys)}, which conditions on \LC{elem x ys} and then fills with \LC{else} case with \LC{trivial}.
The last proof macro is \LC{induct xs}, which pattern matches on \LC{xs} into 2 macro-processing branching: \LC{[]} and \LC{x':xs'}.
\footnote{In the implementation, these fresh variables are given unique suffixes, but this unecessarily intrudes on readability.}
In the second branch, the variables \LC{x'} and \LC{xs'} are added to the recursion context for the 2nd argument position, since \LC{xs} is in the 2nd argument position.
Finally, since the last proof macro was \textit{not} an evidence macro, a \LC{auto} is implicitly appended.
This \LC{auto} is processed in each branch resulting from \LC{induct xs}.
In the branch for \LC{[]}, the only values in scope are \LC{x} and \LC{ys}, so there are no neutral forms to generate.
The recursion \LC{elem_concat x ys ys} is not generated because none of the arguments are from the recursion context of their respective argument position.
In the branch for \LC{[]}, the values in sccope are \LC{x}, \LC{x'}, \LC{xs'}, and \LC{ys}, so there are several neutral forms that can be generated -- in particular, several recursions:
\begin{itemize}
  \item \LC{elem_concat x' xs' xs'}
  \item \LC{elem_concat x' xs' ys}
  \item \LC{elem_concat x xs' xs'}
  \item \LC{elem_concat x xs' ys}
\end{itemize}
These are valid recursions because all of them have at least one argument from the recursion context at that argument position: \LC{xs'}.
Altogether, the resulting \LangBTerm is the following:
\begin{code}
  elem_concat :: N -> [N] -> [N] -> Proof
  elem_concat = \x xs ys ->
    if elem x ys then
      case xs of
        Nil -> trivial
        Cons x' xs' -> Auto
          { init = [ elem_concat x' xs' xs' 
                   , elem_concat x' xs' ys
                   , elem_concat x  xs' xs'
                   , elem_concat x  xs' ys ]
          , kept = []
          , pruned = [] }
    else
      trivial
\end{code}
Here, the \LC{Auto} structure corresponds to the term
\begin{code}
  elem_concat x' xs' xs' &&&
  elem_concat x' xs' ys  &&&
  elem_concat x  xs' xs' &&&
  elem_concat x  xs' ys
\end{code}
with the additional metadata that no terms have been \textit{kept} (i.e. determined to be not safely pruned) and \textit{pruned} (i.e. determined to be safely pruned).

Of course, it turns out that not all of these terms are needed for a valid proof.
The pruning process, described in the next subsection, describes how the \LC{Auto} structure's metadata is used statefully to safely prune the unnecessary terms.
  
\subsection{Pruning}

\subsubsection{Linear Pruning}

For each \LC{Auto} structure in a \LangBTerm, each \textit{exp} in its \LC{init} field is attempted to be pruned one at a time.
This is done by moving the \textit{exp} from the \LC{init} field to the \LC{pruned} field, embedding and splicing the new \LangBTerm into the original Haskell file in place of the original proof macro, and then running Liquid Haskell to check if this prune was safe.
If it was then pruning continues with the rest of the \textit{exp}s in the \LC{init} fields of the \LC{Auto} structures, otherwise this prune is undone, and the \textit{exp} that was attempted to be pruned is instead moved to the \LC{kept} field before continuing pruning.

Recall the \LangBTerm that resulted from processing the proof macro used to prove \LC{elem_concat}.
There is an \LC{Auto} structure that clearly has a few \textit{exp}s that can be pruned since they are unnecessary for the proof.
The subset of necessary \textit{exp}s is found via the linear pruning procedure, trying to remove each \textit{exp} one at a time to see which can be safely removed.
After pruning, the final resulting \LangBTerm can be embedded into Haskell a last time and presented to the user as a valid proof:

\begin{code}
  elem_concat :: N -> [N] -> [N] -> Proof
  elem_concat = \x xs ys ->
    if elem x ys then
      case xs of
        Nil -> trivial
        Cons x' xs' -> elem_concat x xs' ys
    else
      trivial
\end{code}

\todo{onclusion of Linear Pruning subsection}

\todo{pretty sure this shouldn't be included}
\subsubsection{Superlinear Pruning}

Since the set of \textit{exp}s that can be safely pruned is an arbitrary subset of all the \textit{exp}s in \LC{Auto} structures in a \LangBTerm, given the information available at this stage, it is impossible to achieve generally superlinear pruning runtime. 
However, very often the size of the subset of \textit{exp}s that need to be kept is constant size -- around 1-4.
So in practice it may be feasible to perform a sort of binary search.
\todo{describe more?}

\section*{Temporary}

\todo{where to put this?}
Note that, for Liquid Haskell, the refinement of the type of \LC{f x y, trivial)} is the same as the refinement of the type of \LC{f x y}.
So, the refinement of the expression's type can be included into the refinement context by simply adding it into the resulting Haskell expression

% \todo{where to put this? i have a different example i used for this section, so do I need to bring back \LC{assoc\_min} again?}
The proof of \LC{assoc\_min} can be expressed shortly as follows:
  
 The "induct a" macro does case analsis on "a", resulting in two branches. Since both branches are to be handled, the following two macros are executed in both branches (matching the behavior of the ";" tactic combinator in Coq).
 Additionally, in the "S a'" case, the "induct a" tactic notices that "a'" was introduced by inducting on an input variable term in positon 0, so "a'" is marked as a valid argument to a recursive call to "assoc\_min" in position 0.
 The same happens for "induct b" and "induct c", yielding the same branches of the verbose Haskell term.
 Finally, every proof macro implicitly ends with "auto" unless it ends with "trivial" or "use e".
 So, in each of the 9 branches, "auto" is executed to generate all well-typed terms of type "Proof" using things in local context.
 In all but the "S a', S b', S c'" case, there are no such well-typed applications, since the only way to produce a term of type "Proof" is to use "assoc\_min", but only in that final is there a term in context that is available to use as the argument to a recursive call to "assoc\_min", for each of its argment positions (which are 0, 1, 2 since it has 3 arguments and each is inducted on).
  