\section{Implementation}

Organization idea:
\begin{enumerate}
  \item Overview the entire procedure:
  \begin{enumerate}
    \item The user writes a proof macro term in ``tactic'' quasiquotes as their proof, in EDSL1 i.e. the proof macro language
    \item The proof macro term is preprocessed into a ``proto proof term'' in EDSL2 i.e. the metadata-augmented subset of Haskell relevant for extrinsic proofs
    \item The ``proto proof term'' is embedded into Haskell, then spliced the place of the original proof macro. The file is checked.
    \item If the check passes, then the proto proof term is pruned by repeatedly pruning the proto proof term, embedding and splicing it into the place of the original proof macro, and checking if the pruned result still passes.
    \item The resulting proof term is a valid proof as determined by Liquid Haskell.
  \end{enumerate}
  \item Detail the features of EDSL1 and the intended behaviors
  \item Briefly describe the details of EDSL2
  \begin{enumerate}
    \item Describe how each feature in EDSL1 is implemented by a preprocessing transformation into EDSL2
    \item Preprocessing EDSL1 into EDSL2 is a contextual transformation, since it needs to keep track of an environment which includes various information (see Language.Core.Syntax.Environment'')
    \item How the ``auto'' term in EDSL2 includes metadata for which terms have have been pruned and which have been determined necessary.
  \end{enumerate}
  \item Detail pruning procedure
  \begin{enumerate}
    \item Clearly, this can be done more efficiently.
    \item Point out further possible work like checking for well-typed applications where the function has refined argument types.
  \end{enumerate}
\end{enumerate}

% 
% * real start
% 
  
The proof macro system processes an input proof macro into an output Liquid Haskell proof term into stages:
\begin{enumerate}
  \item 
  The user writes the input proof macro in Template Haskell quasiquotes. 
  Then the user runs \TheTool \todo{how to talk about this executable} on the file containing the target proof macro to process. 
  The quasiquoted proof macro is parsed into \LangA.
  \item
  The \LangATerm is processed into a corresponding \LangBTerm, and all metadata values are given initial defaults.
  \item
  The \LangBTerm is cached, embedded into Haskell, and then spliced in place of the original proof macro. 
  \item 
  The cached \LangBTerm is repeatedly pruned, using metadata to track pruning progress, where each pruning step involves removing some proof terms contained inside of the entire \LangBTerm, embedding and splicing it in place of the original proof macro, and then running Liquid Haskell to check that the prune step was valid.
\end{enumerate}
  
\subsection{The \LangA}

The \LangA defines a collection of proof macros that are meant to resemble Coq tactics.
The proof macro system is designed to be extendible, so that new proof macros can be added easily by adding a new constructor to the \LangA and then handling the new case for preprocessing.
The syntax for \LangA is the following:

\begin{align*}
  \textit{decl-macro} ::= &
    f ~ : ~ \textut{typ} \\ &
    f ~ \overline{y_i} ~ = ~ \overline{\textit{exp-macro} ;}
  \\[1em]
  \textit{exp-macro} ::= &
    \TC{induct} ~ x \\ | &
    \TC{destruct} ~ \textit{exp} \\ | &
    \TC{assert} ~ \textit{exp} \\ | &
    \TC{dismiss} ~ \textit{exp} \\ | &
    \TC{condition} ~ \textit{exp} \\ | &
    \TC{auto} ~ [\overline{\textit{x} ,}] ~ n \\ | &
    \TC{use} ~ \textit{exp} \\ | &
    \TC{trivial}
  \\[1em]
  \textit{exp} ::= \text{Haskell expression} \\
  \textit{typ} ::= \text{Haskell type (monomorphic)} \\
  \textit{f, x, y_i} ::= \textit{Haskell name} \\
  \textit{n} \in \mathbb{N}
\end{align*}

There are two main types of proof macros: 
\begin{itemize}
  \item
  \textit{Control flow} macros are processed into control flow structures, such as pattern matching. 
  If such a macro yields a control flow structure that has mutliple branches, then a macro-processing branch is created for each of these branches, and the sequence of proof macros following it are processed in each of these branches.
  The control flow macros are:
  \begin{itemize}
    \item
    $\TC{induct} ~ x$ --- patterm matches on $x$, which must be a function argument in the declaration macro.
    In each case of the pattern match, the introduced variables are included in the recursion context at the corresponding argument position of $x$.
    This recursion context keeps track of what expressions are allowed to be given as an argument, in that argument position, to a recursion.
    A macro-processing branch is created for each case.
    Note that the hints and height arguments to this macro are optional, and take values \code{[]} and \TC{3} respectively by default.
    \item $\TC{destruct} ~ \textit{exp}$ --- pattern matches on $\textit{exp}$. A macro-processing branch is created for each case.
    \item $\TC{condition} ~ \textit{exp}$ --- conditions on $\text{exp}$. A macro-processing branch is created for the \TC{then} and \TC{else} branches respectively.
    \item $\TC{assert} ~ \textit{exp}$ --- conditions on $\text{exp}$. A macro-processing branch is created for the \TC{then} branch, but the \TC{else} branch is only filled with \TC{trivial}.
    \item $\TC{dismiss} ~ \textit{exp}$ --- conditions on $\text{exp}$. A macro-processing branch is created for the \TC{else} branch, but the \TC{then} branch is only filled with \TC{trivial}.
  \end{itemize}
  \item
  \todo{think of better name? perhaps \textit{evidence-providing}?}
  \textit{Evidence} macros are processed into terms that provide evidence to the Liquid Haskell checker, such as introducing a lemma to the refinement context.
  \todo{is this the right place to put this note?}
  Note that if a sequence of macros given in declaration macro (i.e. \textit{decl-macro}) does \textit{not} end in an evidence macro, then an \TC{auto} macro is included implicitly at the end of the sequence.
  The evidence macros are:
  \begin{itemize}
    \item
    $\TC{auto} ~ [\overline{\textit{x} ,}] ~ n$ --- generates all well-typed neutral forms that have type \TC{Proof}, up to height $n$, using variables in context and given as hints. 
    The height of a neutral form is the height of its AST, where applications are multi-ary rather than nested binary (e.g. \TC{f x y z} has height 2 rather than height 4).
    Recursive neutral forms are also generated in this way, but have the additional restriction that a recursive neutral form must have as one of its arguments a variable from that argument position's recursion context, ensuring that Liquid Haskell will be able to determine that the resulting Haskell function is terminating.
    \item $\TC{use} ~ \textit{exp}$ --- includes the refinement of the expression's type into the refinement context.
    \item $\TC{trivial}$ --- includes \TC{trivial :: Proof} into the resulting Haskell term. Note that \TC{trivial = ()} and \TC{Proof = ()}.
  \end{itemize}
\end{itemize}

% TODO: for sake a readability, I used the usual list notations such as [N] and (++). is this ok? since in the actual implementation, I decided not to overload the built-in Haskell list notations
For example, the following is a proof macro for generating a proof of ``if \LC{x} is in a list \LC{xs}, then \LC{x} is also in the list \LC{xs ++ ys}'':
\begin{code}
  {-@ 
  elem_concat ::
    x:N -> xs:[N] -> ys:[N] ->
    {elem x ys ==> elem x (xs ++ ys)}
  @-}
  [tactic|
  elem_concat :: N -> [N] -> [N] -> Proof
  elem_concat x xs ys =
    induct xs;
    assert {elem x ys}
  |]
\end{code}


% 
% TODO: all text below this is temporary
% 

% TODO: put this in the implementation details section:
Note that, for Liquid Haskell, the refinement of the type of \TC{f x y, trivial)} is the same as the refinement of the type of \TC{f x y}.
So, the refinement of the expression's type can be included into the refinement context by simply adding it into the resulting Haskell expression

 The proof of "assoc\_min" can be expressed shortly as follows:
  
 The "induct a" macro does case analsis on "a", resulting in two branches. Since both branches are to be handled, the following two macros are executed in both branches (matching the behavior of the ";" tactic combinator in Coq).
 Additionally, in the "S a'" case, the "induct a" tactic notices that "a'" was introduced by inducting on an input variable term in positon 0, so "a'" is marked as a valid argument to a recursive call to "assoc\_min" in position 0.
 The same happens for "induct b" and "induct c", yielding the same branches of the verbose Haskell term.
 Finally, every proof macro implicitly ends with "auto" unless it ends with "trivial" or "use e".
 So, in each of the 9 branches, "auto" is executed to generate all well-typed terms of type "Proof" using things in local context.
 In all but the "S a', S b', S c'" case, there are no such well-typed applications, since the only way to produce a term of type "Proof" is to use "assoc\_min", but only in that final is there a term in context that is available to use as the argument to a recursive call to "assoc\_min", for each of its argment positions (which are 0, 1, 2 since it has 3 arguments and each is inducted on).
  
\subsection{Preprocessing}
  
%s EDSL1 is used by the user to write proof macros inside of quasiquotes of the form \texttt{[tactic|...|]}
%  
% %Grammar of EDSL1:
% \begin{align*}
%   \textit{decl}~ ::= &
%     \textit{name} ~ : ~ \textit{type} \\ &
%     \textit{name} ~ \overline{\textit{name}} ~ = ~ \overline{\textit{instr} ; }
%   \\
%    \textit{instr} ~ ::= &
%      \textbf{intro} ~ \textit{name} \\ | &
%      \textbf{destruct} ~ \textit{name} ~ (\textbf{as} ~ \textit{destruct-pat}) ([\overline{\textit{flag},}]) \\ | &
%      \textbf{induct} ~ \textit{name} ~ (\textbf{as} ~ \textit{pat}) ([\overline{\textit{induct-flag},}]) \\ | &
%      \textbf{auto} ~ (\overline{\textit{name},}) (\textit{nat}) \\ | &
%      \textbf{condition} ~ \textit{exp} ~ (\textbf{requires} ~ [\overline{\textit{name},}]) \\ | &
%      \textbf{assert} ~ \textit{exp} ~ (\textbf{requires} ~ [\overline{\textit{name},}]) \\ | &
%      \textbf{dismiss} ~ \textit{exp} ~ (\textbf{requires} ~ [\overline{\textit{name},}]) \\ | &
%      \textbf{use} ~ \textit{exp} ~ (\textbf{requires} ~ [\overline{\textit{name},}]) \\ | &
%      \textbf{trivial}
%  \end{align*}
%   
 The language EDSL1, defined above, is preprocessed into EDSL2 which is a subset of Haskell augmented with metadata. 
 The conceptual distinction between EDSL1 and EDSL2 is that preprocessing EDSL1 requires a context and producing many expressions per single EDSL1 instruction, whereas EDSL2 is strictly an embedding into Haskell and so each EDSL2 expression corresponds to exactly one Haskell expression.
  
% Grammar of EDSL2:
% \begin{align*}
%   \textit{decl}~ ::= &
%     \textit{name} ~ : ~ \textit{type} \\ &
%     \textit{name} ~ \overline{\textit{name}} ~ = ~ \textit{expr}
%   \\
%   \textit{expr}~ ::= &
%     \mathbf{\lambda} ~ \textit{name} ~ \textbf{->} ~ \textit{instr} \\ &
%     \textbf{case} ~ \textit{exp} ~ \textbf{of} ~ \overline{\textit{pat} ~ \textbf{->} ~ \textit{exp} ;} \\ & 
%     \textbf{if} ~ \textit{exp} ~ \textbf{then} ~ \textit{exp} ~ \textbf{else} ~ \textit{exp} \\ & 
%     \textbf{auto} ~ [\overline{\textit{exp} ;}] \textit{pruning-metadata} \\ & 
%     \textbf{trivial}
% \end{align*}
%  
 An EDSL1 declaration is preprocessed into an EDSL2 declaration like so:
  
 \begin{verbatim}
 preprocessDecl : EDSL1-decl -> EDSL2-decl
 preprocessDecl (name, args, type, instrs) =
   set function name to name
   set output type to type
   for each arg
     get output type as a -> b 
     store type of arg as a
     set output type to b
   preprocess instrs
  
 // stateful
 preprocess : [EDSL1-instr] -> EDSL2-expr
 preprocess [] = []
 preprocess (instr : instrs) = case instr of 
   intro x -> do
     [|a <- b|] <- get output type
     type of x := a
     output type := b
     [|\x -> $(preprocess instrs)|]
   destruct exp requires xs -> do
     if all of xs are in scope then 
       varss <- get patterns for deconstructing exp
       constrs <- constructors of the datatype of exp
       matches <- 
         for each (vars, constr) in (varss * constrs)
           args <- argument types of constr
           for each (var, arg) in in (pats * args)
             type of var := arg
           [|ps -> $(preprocess instrs)|]
       [|case exp of matches|]
     else
       preprocess instrs
   induct exp requires xs -> do
     if all of xs are in scope then 
       varss <- get patterns for deconstructing exp
       constrs <- constructors of the datatype of exp
       matches <-
         for each (vars, constr) in (varss * constrs)
           args <- argument types of constr
           for each (var, arg) in in (pats * args)
             if exp is an argument to the top-level proposition then
               add var to context of recursive-safe expression to fill the
               respective argument of a recursive call
             type of var := arg
           [|ps -> $(preprocess instrs)|]
       [|case exp of matches|]
     else
       preprocess instrs
   auto hints depth ->
     generate all neutral forms in context, up to depth
     when a recursion is generated, at least one of its arguments must be something from the context of recursive safe expressions for that argument (which is populated via induct)
   assert exp requires xs ->
     if all of xs are in scope then
       [|if exp then $(preprocess instrs) else True|]
     else
       preprocess instrs
   dismiss exp requires xs ->
     if all of xs are in scope then
       [|if exp then True else $(preprocess instrs)|]
     else
       preprocess instrs
   use exp requires xs -> 
     [|use exp &&& $(preprocess instrs)]
   condition exp requires xs ->
     [|if exp then $(preprocess instrs) else $(preprocess instrs)]
   trivial ->
     [|trivial|]
 \end{verbatim}

 %  TODO: use running example, simpler than min_assoc -- something that uses induct, auto (with lemma hints), cond

Consider the following example proof macro:
% {-@ reflect prop @-}
% prop x xs ys = 
%   if elemListN x ys then 
%     elemListN x (concatListN xs ys)
%   else 
%     True
%  
% {-@ automatic-instances proof @-}
% {-@
% proof :: x:N -> xs:ListN -> ys:ListN -> {prop x xs ys}
% @-}
% -- elemListN x1 (concatListN (Cons x2 xs) ys)
% -- elemListN x1 (Cons x2 (concatListN xs ys))
% -- if x1 == x2 then 
% --   QED
% -- else
% --   elemListN x1 (concatListN xs ys)
% -- [tactic|
% -- proof :: N -> ListN -> ListN -> Proof
% -- proof x xs ys =
% --   induct xs;
% --   condition {elemListN x ys}
% -- |]
% -- %tactic:begin:proof
% proof :: N -> ListN -> ListN -> Proof
% proof = \x -> \xs -> \ys -> case xs of
%                                 Data.Nil -> if elemListN x ys then trivial else trivial
%                                 Data.Cons n_0 listN_1 -> if elemListN x ys
%                                                           then proof x listN_1 ys
%                                                           else trivial
The following procedure describes how \LC{elem_concat} is processed into \LangB:
\begin{enumerate}
  \item 
\end{enumerate}
  
\subsection{Pruning}

\subsubsection{Linear Pruning}

For each auto site (i.e. each apreprocearance of \textbf{auto} in the preprocessed EDSL2), each \textit{exp} it uses is pruned one at a time, checking to see if Liquid Haskell still passes after the pruning. If a prune succeeds the pruned \textit{exp} is discarded, otherwise the pruned \textit{exp} is returned to the original \textbf{auto} site.
