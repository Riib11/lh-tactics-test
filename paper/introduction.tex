\section{Introduction}

Liquid Haskell~\cite{liquidHaskell} is a popular verifier for Haskell
programs, leveraging the power of SMT solvers~\cite{BarST-RR-10} (such
as Z3~\cite{Z3} or CVC4~\cite{CVC4}) to prove the correctness of
diverse applications ranging from optimizations~\cite{TPE2018} to
string matching algorithms~\cite{TaleOfTwoProvers}. Specifications for
these applications are written in the form of {\em refinement
  types}~\cite{RefinementForML}, boolean predicates over program
values.

For concreteness, consider the following \LC{min} function that
computes the minimum of two natural numbers, defined inductively:
\begin{code}
  data N = Z | S N
  
  min :: N -> N -> N 
  min Z _ = Z
  min _ Z = Z
  min (S m) (S n) = S (min m n)
\end{code}

\newcommand{\imin}{\textit{min}~}
Naturally, we would expect such a function to be associative, that is:
$$ \forall a ~ b ~ c. ~ \imin (\imin a~b)~c == \imin a~(\imin~b~c) $$
%
In Liquid Haskell, we can {\em specify} associativity by defining a
refinement type to encode this property, and we can {\em prove}
associativity by defining a term of that type:
%
\begin{code}
  {-@ assoc_min :: a:N -> b:N -> c:N ->
      {_:() | min (min a b) c == min a (min b c)}
  @-}
  assoc_min :: N -> N -> N -> Proof
  assoc_min a b c = ...
\end{code}
%
To Haskell, the type of \LC{assoc_min} is simply a function with three
natural number arguments that returns a \LC{Proof}, which is just a
type synonym for \LC{()}. To Liquid Haskell, however, the type of
\LC{assoc_min} is much more interesting: its return type does not only
specify that the output is a unit, but {\em refines} it so that
associativity of \LC{min} holds for its input arguments. In other
words, the only interesting thing about the result of this function is
its refinement, which constitutes an ``extrinsic style'' proof of
associativity. This is a common enough pattern that Liquid Haskell
supports dropping the ``\LC{_:()}'' part of the refinement for
brevity, as we will also do in the remainder of this paper.

But how does Liquid Haskell decide if the refinement type is true? By
reducing typechecking to verification conditions that SMT solvers
reason about. However, while SMT solvers are pre-programmed with a
wide assortment of facts about various domains such as integer
arithmetic and boolean logic, they don't really know anything about
user-defined data types like \LC{N} or user-defined functions like
\LC{min}. While a direct encoding of such features to SMT is possible
in principle~\cite{HALO}, it unfortunately leads to unpredictable
verification, also known as the ``butterfly
effect''~\cite{LeinoP16}. To that end, Liquid Haskell lifts
user-defined data types and functions into a representation that can
be handled symbolically by SMT solvers~\cite{VazouTCSNWJ18}. Still,
many true properties of user-defined data types and functions are
still not automatically verifiable: users must guide, via refined
Haskell code, the SMT solver to simpler cases that can be checked
automatically.

Unfortunately, given the lack of interactivity of Liquid Haskell, it
is not always clear what the gap in understanding between the user and
the SMT solver is, which often makes writing such refined code a
tedious and frustrating process. Consider again associativity for the
\LC{min} function. On paper, we can informally reason that
associativity holds by induction on the natural numbers that are inputs
to \LC{min}, due to its simple recursive structure. In Liquid Haskell,
the refined code that finally convinces the SMT solver that the
program typechecks is shown in Figure~\ref{fig:assoc-min-proof}.


% The SMT solver, however, does not know about induction (unlike Coq where a tactic "induction" could solve this goal right away).
% And, because of Liquid Haskell's limited capabilities for quantifying over predicates (TODO: include reference here), it is unfeasible to write the induction principle for natural numbers as a refined Haskell function.
% So, a proof of "assoc\_min" is written as follows:
\begin{figure}
\begin{code}
  {-@ assoc_min :: a:N -> b:N -> c:N ->
        {min (min a b) c == min a (min b c)} @-}
  assoc_min :: N -> N -> N -> Proof
  assoc_min a b c =
    case a of 
      Z ->
        case b of 
          Z ->
            case c of
              Z -> trivial
              S c' -> trivial
          S b' ->
            case c of
              Z -> trivial
              S c' -> trivial
    S a' ->
      case b of 
        Z ->
          case c of
            Z -> trivial
            S c' -> trivial
        S b' ->
          case c of
            Z -> trivial
            S c' -> assoc_min a' b' c'
\end{code}
\caption{Liquid Haskell proof term for associativity of \texttt{min}}
\label{fig:assoc-min-proof}
\end{figure}

{\em All} of the branches of pattern matching on \LC{a}, \LC{b}, and
\LC{c} must be written out explicitly. Otherwise, the SMT solver would
not know how to simplify the \LC{min} expressions in the
refinement---the only facts it knows are the three equations that were
used in \LC{min}'s definition:
%
\LC{min Z _ = Z}, \LC{min _ Z = Z}, and
%
\LC{min (S x) (S y) = S (min x y)}.
%
Liquid Haskell understands the constraints introduced by pattern
matching, and takes them into account in order to discharge most
cases---the non-recursive ones that involve at least one \LC{Z}. The
proof conclusion in such cases is \LC{trivial}, which is again just a
synonym for the term-level \LC{()}.

However, in the recursive case of \LC{min}, the Liquid Haskell typechecker needs
additional help, in the form of a recursive call to \LC{assoc_min a' b' c'}, which
brings its refinement in scope for the SMT solver and allows it to conclude that
the induced verification condition holds. Crucially, this refinement is again the
only thing that matters: while the structure of the term gives the appearance
of a proof term in the style of Coq or Agda, the actual return value doesn't matter.
We could just as well have written something like
\begin{code}
  snd (assoc_min a' b' c', ())
\end{code}
and Liquid Haskell would still gladly accept the definition. In fact,
Liquid Haskell's conjunction operator (\LC{&&&}) is defined exactly
this way: it takes two \LC{Proof}s and returns the second one---its
only effect is making the refinement of both arguments visible to the SMT solver.

Even in this simple example of associativity of \LC{min}, the full
verbosity required is cumbersome and obscures the fact that the
underlying argument is a straightforward induction. In larger
developments where the SMT solver might need to rely on helper lemmas,
this problem only becomes more pronounced.  Other proof assistants,
such as Coq~\cite{Coq}, Lean~\cite{Lean4}, or
Isabelle~\cite{Isabelle}, rely on interactive tactics in these
situations to aid users' proof efforts. But developers of these tactic
languages enjoy a transparent API to interact with the current proof
state, and an essentially clean slate to design metaprogramming
capabilities, which has been exploited to the great benefit of proof
assistant users~\cite{Ltac, Mtac, ltac2}.

Unfortunately, Liquid Haskell interacts with the SMT solver in a very
opaque manner, and within the Haskell ecosystem metaprogramming
capabilities are already well established in the form of Template
Haskell---but not designed specifically with SMT-based verification in
mind. So then, {\em what can we do within the confines of this mature
  Haskell ecosystem to aid users?}  Without interactivity, an
interface to concise proof generators must expand to a proof term all
at once i.e. it must behave like a macro. Therefore, we developed a
macro system for generating Liquid Haskell proof terms, using the
existing metaprogramming tools for Haskell.
  % TODO: any other limitations to mention, that directed us to this solution?

\paragraph*{Liquid Proof Macros}

In this paper, we show how to leverage the power of Template Haskell
to automate proof term generation for Liquid Haskell.  We develop {\em
  Liquid Proof Macros}, an extensible DSL in which users can write
intutive proofs that resemble automated tactics%
\footnote{We refrain calling our DSL tactics, as that suggests a
  notion of interactivity that is impossible in the current version
  of Liquid Haskell.}%
%
 of more traditional proof assistants, including case analysis,
 induction, conditioning, and proof search. For example, the same
 proof of associativity of \LC{min} using Liquid Proof macros can be
 seen in Figure~\ref{fig:assoc-min-macro}.

These macros are expanded to a subset of Haskell that resembles, or
rather is even more complicated than, the one used in
Figure~\ref{fig:assoc-min-proof}. To facilitate typechecking of larger
Liquid Haskell developments, we also augment this subset with metadata
information, and provide a pruning algorithm reminiscent of shrinking
in property-based testing~\cite{ClaessenH00}, simplifying away any
unnecessary components that result from proof search.


\begin{figure}[t]
\begin{code}
  {-@ assoc_min :: a:N -> b:N -> c:N ->
        {min (min a b) c == min a (min b c)} @-}
  [tactic|
    assoc_min :: N -> N -> N -> Proof
    assoc_min a b c = induct a; induct b; induct c
  |]
\end{code}
\caption{Associativity of \texttt{min} using Liquid Proof Macros}
\label{fig:assoc-min-macro}
\end{figure}

\pagebreak
In this paper we make the following contributions:
\begin{itemize}
\item We describe a methodology for using Template Haskell to
  automatically construct Liquid Haskell proof terms, and we develop
  an extensible framework using this methodology for automating
  inductive proofs in Liquid Haskell
  (Section~\ref{sec:liquid-proof-macros}).
\item We evaluate our framework against an existing benchmark
  containing a wide variety of Liquid Haskell properties, and found
  that our Liquid Proof Macros can be used to automate all of these
  properties, leading to a $1.57\times$ reduction in code on average
  (Section~\ref{sec:eval})
\end{itemize}
We then discuss related work (Section~\ref{sec:related}), before
concluding with a discussion of the limitations of our framework and
directions for future work (Section ~\ref{sec:future}).
