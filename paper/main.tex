%% The first command in your LaTeX source must be the \documentclass command.
\documentclass[sigplan,screen]{acmart}

\newif\ifdraft\drafttrue
%% Notes

\usepackage{xcolor}
\usepackage{xspace}

\definecolor{darkgreen}{RGB}{0,150,0}
\definecolor{darkorange}{RGB}{191,114,13}
\newcommand\todo[1]{\ifdraft\textcolor{red}{\textbf{TODO:} #1}\fi}
\newcommand\nv[1]{\ifdraft\textcolor{purple}{\textbf{NV:} #1}\fi}
\newcommand\mmg[1]{\ifdraft\textcolor{teal}{\textbf{MMG:} #1}\fi}
\newcommand\leo[1]{\ifdraft\textcolor{darkgreen}{\textbf{LEO:} #1}\fi}
\newcommand\hen[1]{\ifdraft\textcolor{darkorange}{\textbf{HENRY:} #1}\fi}

\newcommand{\cn}{\ifdraft\textsuperscript{\textcolor{blue}{[citation needed]}}\xspace\fi}

%% Text 
\newcommand\ie{{i.e.,}\xspace}
\newcommand\eg{{e.g.,}\xspace}
\newcommand\resp{{resp.}\xspace}

%% System names
\newcommand\Fstar{F${}^*$}

%% EDSL names
\newcommand{\LangA}{proof macro language\xspace}
\newcommand{\LangATerm}{proof macro\xspace}
\newcommand{\LangB}{proto-proof language\xspace}
\newcommand{\LangBTerm}{proto-proof term\xspace}

\newcommand{\TheTool}{the tool\xspace}

\usepackage{listings}

% uncomment next line to restore colors
\def\withcolor{}


\ifdefined\withcolor
  \definecolor{fstarblue}{rgb}{0.0, 0.0, 1.0}
  \definecolor{haskellstr}{rgb}{0.2, 0.2, 0.6}
  \definecolor{haskellred}{rgb}{1.0, 0.0, 0.0}
  \definecolor{gray_ulisses}{gray}{0.55}
  \definecolor{castanho_ulisses}{rgb}{0.59,0.42,0.15}
  \definecolor{preto_ulisses}{rgb}{0.55,0.28,0.59}
  \definecolor{green_ulises}{rgb}{0.59,0.42,0.15}
\else
	\definecolor{fstarblue}{gray}{0.1}
	\definecolor{haskellstr}{gray}{0.1}
	\definecolor{haskellred}{gray}{0.1}
	\definecolor{gray_ulisses}{gray}{0.1}
	\definecolor{castanho_ulisses}{gray}{0.1}
	\definecolor{preto_ulisses}{gray}{0.1}
	\definecolor{green_ulisses}{gray}{0.1}
\fi

\def\codesize{\small}

\lstdefinelanguage{HaskellUlisses} {
	basicstyle=\ttfamily\codesize,
	sensitive=true,
	%% morecomment=[s][\color{gray_ulisses}\ttfamily\itshape\codesize]{-}{-},
	%% morecomment=[l][\color{gray_ulisses}\ttfamily\itshape\codesize]{--},
	%% morecomment=[s][\color{gray_ulisses}\ttfamily\itshape\codesize]{\{-}{-\}},
	%% morecomment=[s][]{\{-@}{@-\}},
	morestring=[b]",
	stringstyle=\color{haskellstr},
	basewidth={0.53em},
	showstringspaces=false,
	numberstyle=\codesize,
	numberblanklines=true,
	showspaces=false,
	breaklines=true,
	showtabs=false,
	tabsize=4,
    literate={ {/\\}{{$\land$}}2
             {->}{{$\rightarrow$}}2
			 {<=>}{{$\Leftrightarrow$}}1
%			 {<=}{{$\leq$}}1
%			 {>=}{{$\geq$}}1
             {forall}{{$\forall$}}1
			 {'a}{{$\alpha$}}1
			 {labelty}{{$l$}}1
             {True}{{$\top$}}1
             {~int}{{$\mathbb{Z}$}}1
             {~nat}{{$\mathbb{N}$}}1
			 {==>}{{$\Longrightarrow$}}1
			 {=>}{{$\Rightarrow$}}1
			 {`feq`}{{$\eqinfix$}}1
			 {ka}{{k${}_a$}}1
			 {kb}{{k${}_b$}}1
			 {dollar}{{$\$$}}1
			 {dsl}{{d$_{sl}$}}2
			 {dfs}{{d$_{fs}$}}2
			 {rsl}{{r$_{sl}$}}2
			 {rfs}{{r$_{fs}$}}2
			 {dlm}{{d$_{lm}$}}2
           },
	emph=
	{[1] Set, Level, Axiom, Propositional, Extensionality, Tot, Type, bool, Lemma, ensures, requires, Ifc, IFC, IfcClearance, GlobalInt, GTot
	},
	emphstyle={[1]\color{fstarblue}},
	emph=
	{[2] class, match, with, if, then, else, let, rec, type, val, in, instance, data, measure, where, effect,noeq, private
	},
	emphstyle={[2]\color{castanho_ulisses}},
	emph=
	{[3]
        lattice, value, equals, canFlow, meet, join, bottom, top, 
        lawBot, lawFlowReflexivity, lawFlowAntisymetry, lawFlowTransitivity, 
        lawMeet, lawJoin, labels, 
        lt, lmeet, ljoin, lcanFlow, eq,
        labeled, labeledTCB
	},
	emphstyle={[3]\color{preto_ulisses}\textbf},
	emph=
	{[4]
        Low, Medium, High
	},
	emphstyle={[4]\color{green_ulises}\textbf},
	emph=
	{[5] assume, admit, admitP
	},
	emphstyle=[5]\color{red}\textbf,%\underline, % underline not working
	% this emp 6 is a ban rule => highlight bad code
	emph={[6] leq, equals, join', c_0, c_1
	},
	emphstyle=[6]\color{green}\textbf,
}

\newcommand{\LC}{\lstinline[language=HaskellUlisses, basicstyle=\ttfamily]}

\lstnewenvironment{code}
{\lstset{language=HaskellUlisses}}
{}

\lstnewenvironment{scode}
{\lstset{language=HaskellUlisses,basicstyle=\ttfamily\footnotesize,keepspaces,mathescape}}
{}


\lstnewenvironment{mcode}
{\lstset{language=HaskellUlisses,columns=fullflexible,keepspaces,mathescape}}
{}

\lstnewenvironment{ccode}
{\lstset{language=C,columns=fullflexible,keepspaces,mathescape}}
{}

\lstMakeShortInline[language=HaskellUlisses,mathescape,keepspaces,mathescape,basicstyle=\ttfamily\codesize,breakatwhitespace]|


\usepackage[inference]{semantic}

\newcommand{\Rho}{\mathrm{P}}
\newcommand{\macroHole}[2]{\Box_{#1 ; #2}}
\newcommand{\expandsTo}{\rightsquigarrow}

\newcommand{\Expand}[4]{#1; #2 \vdash #3 \expandsTo #4}
\newcommand{\Interp}[2]{\llbracket #1 \rrbracket (#2)}
\newcommand{\Sem}[3]{\Interp{#1}{#2} = #3}

%% Rights management information.  This information is sent to you
%% when you complete the rights form.  These commands have SAMPLE
%% values in them; it is your responsibility as an author to replace
%% the commands and values with those provided to you when you
%% complete the rights form.
\setcopyright{none}
\copyrightyear{2018}
\acmYear{2018}
\acmDOI{XXXXXXX.XXXXXXX}

%% These commands are for a PROCEEDINGS abstract or paper.
% \acmConference[Conference acronym 'XX]{Make sure to enter the correct
%   conference title from your rights confirmation emai}{June 03--05,
%   2018}{Woodstock, NY}
% \acmPrice{15.00}
% \acmISBN{978-1-4503-XXXX-X/18/06}

%%
%% The majority of ACM publications use numbered citations and
%% references.  The command \citestyle{authoryear} switches to the
%% "author year" style.
%%\citestyle{acmauthoryear}

\begin{document}

\title{Liquid Proof Macros}

\author{Henry Blanchette}
\email{blancheh@umd.edu}
\affiliation{%
  \institution{University of Maryland}
  \city{College Park}
  \country{USA}
}
%\orcid{1234-5678-9012}

\author{Niki Vazou}
\email{niki.vazou@imdea.org}
\affiliation{%
  \institution{IMDEA}
  \city{Madrid}
  \country{Spain}
}

\author{Leonidas Lampropoulos}
\affiliation{%
  \institution{University of Maryland}
  \city{College Park}
  \country{USA}
}
\email{leonidas@umd.edu}

%%
%% By default, the full list of authors will be used in the page
%% headers. Often, this list is too long, and will overlap
%% other information printed in the page headers. This command allows
%% the author to define a more concise list
%% of authors' names for this purpose.
\renewcommand{\shortauthors}{Blanchette et al.}

%%
%% The abstract is a short summary of the work to be presented in the
%% article.
\begin{abstract}
  Proof Macros for Liquid Haskell
\end{abstract}

%%
%% The code below is generated by the tool at http://dl.acm.org/ccs.cfm.
%% Please copy and paste the code instead of the example below.
%%
\begin{CCSXML}
<ccs2012>
   <concept>
       <concept_id>10011007.10011074.10011099.10011692</concept_id>
       <concept_desc>Software and its engineering~Formal software verification</concept_desc>
       <concept_significance>500</concept_significance>
       </concept>
 </ccs2012>
\end{CCSXML}

\ccsdesc[500]{Software and its engineering~Formal software verification}

%%
%% Keywords. The author(s) should pick words that accurately describe
%% the work being presented. Separate the keywords with commas.
\keywords{Liquid Haskell, proof macros, tactics}

%%
%% This command processes the author and affiliation and title
%% information and builds the first part of the formatted document.
\maketitle

\section{Introduction}

Writing Liquid Haskell proofs is hard.
Using Template Haskell we can make it easier!

Our contributions are:
\begin{itemize}
\item A methodology for using Template Haskell to automatically construct Liquid Haskell hints.
\item An instantiation of that framework for automating Liquid Haskell inductive proofs.
\end{itemize}

\section{Extended Example}

\leo{Comments work?}

Sample citation: \cite{liu20typeclasses}

\section{Using Template Haskell for Liquid Proofs}

\section{Implementation}

%%
%% The acknowledgments section is defined using the "acks" environment
%% (and NOT an unnumbered section). This ensures the proper
%% identification of the section in the article metadata, and the
%% consistent spelling of the heading.
\begin{acks}
  We thank Jacob Prinz and the anonymous reviewers for their helpful
  comments.  This work was supported by NSF award \#2107206, {\em
    Efficient and Trustworthy Proof Engineering} (any opinions,
  findings and conclusions or recommendations expressed in this
  material are those of the authors and do not necessarily reflect the
  views of the NSF).
\end{acks}

%%
%% The next two lines define the bibliography style to be used, and
%% the bibliography file.
\bibliographystyle{ACM-Reference-Format}
\bibliography{local}

%%
%% If your work has an appendix, this is the place to put it.
% \appendix

% \section{Proofs}

\end{document}
\endinput
